\documentclass[a4paper]{report}
\usepackage{stmaryrd} % \llbracket, \rrbracket
\usepackage{url}

\title{The CPC manual}
\author{Juliusz Chroboczek, Gabriel Kerneis\\
{\tt <jch@pps.jussieu.fr>, <kerneis@pps.jussieu.fr>}}
\date{20 June 2009}

\begin{document}
\maketitle

\chapter{The CPC language} \label{chapter:language}

CPC is a programming language designed for writing concurrent programs, for
example network servers and clients.  CPC is impemented as a
source-to-source translation from CPC into plain C using a technique known
as {\em translation into Continuation Passing Style\/} (CPS)
\cite{strachey:continuations, plotkin:call-by-lambda}.

The main abstraction provided by CPC is a {\em CPC thread}.  From the
programmer's view, a CPC thread roughly corresponds to other programming
languages' notion of {\em thread\/} or {\em lightweight process}, except
that it has no identity: there is no {\em thread identifier\/} that can be
used to kill or suspend a given CPC thread.

A CPC thread can be in one of two states: {\em attached\/} to the CPC
scheduler, and {\em detached}.  At a given time, the set of all attached
threads are scheduled cooperatively, and an attached thread cannot only be
preempted by other attached threads because of explicit programmer action.
A detached thread, on the other hand, is associated to a native
operating-system thread, and is scheduled by the operating system
asynchronously with respect to all other threads.

\paragraph{Structure of a CPC program}

Just like a plain C program, a CPC program is a set of functions.
Functions in a CPC program are partitioned into ``cps'' functions and
``native'' functions; a global constraint is that a cps function can
only ever be called by another cps function, never by a native
function.  The precise set of contexts where a cps function can be
called is defined in Sec.~\ref{sec:contexts}.

Intuitively, cps code is ``interruptible'': when in the attached state, it
is possible to interrupt the flow of a block of cps code in order to pass
control to another piece of code or to wait for an event to happen.  Native
code, on the other hand, is ``atomic'': if a sequence of native code is
executed in attached state, it must be completed before anything else is
allowed to run.

Technically, native function calls are executed by using the machine's
native stack.  Cps function calls, on the other hand, are executed by
using a lightweight stack-like structure known as a continuation.
This arrangement makes CPC context switches extremely fast; the
tradeoff is that a cps function call is an order of magnitude slower
than a native call.  Thus, computationally expensive code should be
implemented in native code whenever possible.

Execution of a CPC program starts at a native function called {\tt
  main}.  This function usually starts by registering a number of
continuations with the CPC runtime (using {\tt cpc\_spawn},
Section~\ref{sec:cooperating}), and then passes control to the CPC
runtime (by calling {\tt cpc\_main\_loop}, Section~\ref{sec:bootstrapping}).

\section{The CPC language}

CPC is a conservative extension of the 1999 edition of the C programming
language; thus, the syntax of CPC is defined as a set of productions to be
added to the grammar defined in the ISO C99 standard \cite{iso:c99}.

In addition to the reserved words in C99, CPC reserves the words
{\tt cps}, {\tt cpc\_spawn} and {\tt cpc\_attached}.

\subsection{CPS contexts} \label{sec:contexts}

Any instruction, declaration, or function definition in CPC can be in
{\em cps context\/} or in {\em native context}.  Cps context is
defined as follows:
\begin{itemize}
\item the body of a cps function is in cps context
  (Sec.~\ref{sec:cpc-functions};
\item the body of a {\tt cpc\_spawn} statement is in cps context
  (Sec.~\ref{sec:cooperating});
\item the body of a {\tt cpc\_attached} statement is in cps context
  (Sec.~\ref{sec:native-threads}).
\end{itemize}
Any construct that is not in cps context is said to be in native context.

\subsection{CPS functions} \label{sec:cpc-functions}

\[ \mbox{function-specifier} ::= \mathtt{cps} \]

Functions can be declared as being CPS-converted by adding {\tt cps}
to the list of functions specifiers.  The effect of such a declaration
is to put the body of the function in cps context, thus making it
possible to use most of the CPC features.

\[ \mbox{block-item} ::= \mbox{function-definition} \]

Functions can be defined within other functions, as in Algol-family
languages; the inner function can access the variables bound by the
outer one.  Only cps functions can be inner functions, and they must
be within other cps functions.

Free variables of inner functions are copies of the variables of the
enclosing function; thus, a change to the value of the free variable
is not visible in the enclosing function\footnote{Except if the variable
is declared {\tt static}.}.  The free variables are initialized whenever
the inner function is called; thus, their initial value does not depend
on the location of the inner function within the outer one.

\subsection{Bootstrapping} \label{sec:bootstrapping}

\begin{verbatim}
    void cpc_main_loop(void);
\end{verbatim}

\paragraph{\tt cpc\_main\_loop} Since \verb|main| is a native
function, some means is necessary to pass control to cps code.  The
function \verb|cpc_main_loop| invokes the CPC scheduler; it returns
when all threads have been exhausted (i.e.\ where there is
nothing more to do).

\subsection{CPC statements and primitive cps functions}

\[ \mbox{statement} ::= \mbox{cpc-statement} \]

CPC has a number\footnote{Three, to be precise.} of statements not
present in the C language.  Moreover, it provides a set of primitive cps
functions, allowing the programmer to schedule threads and wait for some
events.

It is important to understand that those primitive functions could not
be defined outside the CPC runtime: they must have access to the
internals of the scheduler to operate.  Since they are cps functions,
they are valid only in cps context.

\subsection{Detaching and reattaching} \label{sec:native-threads}

A continuation can be scheduled to be run by a native thread; intuitively,
the continuation ``becomes'' a native thread.  When this happens, we say
that the continuation has been {\em detached\/} from the CPC scheduler,
and {\em attached\/} to some thread pool.  The opposite operation is
known as {\em attaching\/} a detached continuation back to the CPC
scheduler.  It is of course possible to migrate a detached continuation
directly from one thread pool to another.

\begin{verbatim}
    typedef cpc_sched;
    cpc_sched *cpc_default_scheduler, *cpc_default_pool;

    cpc_sched *cpc_threadpool_get(int);
    int cpc_threadpool_release(cpc_sched *);

    cps cpc_sched *cpc_attach(cpc_sched *pool);
    cpc_sched *cpc_get_sched();
\end{verbatim}


\paragraph{\tt cpc\_attach} The {\tt cpc\_attach} statement
attaches the current continuation to the given scheduler.  If the
scheduler is a thread pool, the following statements are executed in a
dedicated native thread, within the thread pool.  If the scheduler is
the special {\tt cpc\_default\_scheduler}, the current continuation is
scheduled by the CPC scheduler.  This function returns a pointer to the
scheduler you are coming from.\\
{\bf Warning:\/} currently, {\tt cpc\_default\_scheduler} is a {\tt
NULL} pointer, but this is {\em not\/} guaranteed to hold in the future.
Always use the {\tt cpc\_default\_scheduler} variable.

\paragraph{\tt cpc\_threadpool\_get} The function {\tt
cpc\_threadpool\_get} returns a pointer to a new thread pool.  It takes
one argument: the maximum number of threads in the pool (thread are
created dynamically).  This argument is cropped to {\tt
MAX\_THREADS}\footnote{{\tt MAX\_THREADS} is an internal constant of the
CPC runtime, currently set to 20.} if it is outside the $\llbracket 1;
\mbox{\tt MAX\_THREADS}\rrbracket$ interval.  This function shall not be
called outside the main loop (it is not thread-safe).

\paragraph{\tt cpc\_threadpool\_release} Conversely, {\tt
cpc\_threadpool\_release} releases a given thread pool.  This function
does not block.  It returns 0 if the pool has been released, -1
otherwise (most notably when they are some threads still detached).  In
that case, you shall retry later --- calling {\tt
cpc\_yield(CPC\_OPPORTUNISTIC)} is recommended (but notice that the flag
is ignored in the current implementation, leading to busy-waiting).
This function can be called in any mode and context, but notice that it
will always fail if you try to release the pool you are attached to.
Conversely, it should always succeed when {\tt cpc\_main\_loop} has
returned.

\paragraph{\tt cpc\_default\_pool} A pointer to the default thread pool,
initialised during the call to {\tt cpc\_main\_loop}.

\paragraph{\tt cpc\_get\_sched} The function {\tt cpc\_get\_sched} returns
the current scheduler.  Although this is not a cps function, it should
be treated as such: this function is valid in cps context only, and no
pointer to it should be taken.

\subsubsection*{Syntactic sugar}
\paragraph{\tt cpc\_attached}
\[ \mbox{cpc-statement} ::=
   \mathtt{cpc\_attached}\ (\mbox{expression})\ \mbox{statement} \]

The body of a {\tt cpc\_attached} statement is run attached to the
scheduler given as its argument: an implicit {\tt s =
cpc\_attach(expression)} is executed upon entering the body, and a {\tt
cpc\_attach(s)} is executed when the end of the statement is reached,
and before any {\tt return} statement inside the body.  Some constructs
are forbidden in the body of a {\tt cpc\_attached} statement because
they disrupt the flow of control: {\tt goto}, {\tt break}, {\tt
continue} and labels.\\
{\em Note:\/} it would be possible to allow the forbidden constructs,
with a smart detection of when one jumps in or out of the body; CPC used
to do this in the past.  But I believe this is ``too much of magic''
with a potentially unhealthy semantics.  Therefore, the programmer has
to take care of such tricky cases by himself.

\paragraph{\tt cpc\_detach, cpc\_detached} Attach to the default thread
pool, if not already detached.
\begin{verbatim}
    #define cpc_detach() \
      cpc_attach(cpc_get_sched() == cpc_default_sched ? \
      cpc_default_pool : cpc_get_sched())
    #define cpc_detached \
      cpc_attached(cpc_get_sched() == cpc_default_sched ? \
      cpc_default_pool : cpc_get_sched())
\end{verbatim}
\subsection{Cooperating: yielding, spawning} \label{sec:cooperating}

\[ \mbox{cpc-statement} ::=
   \mathtt{cpc\_spawn}\ \mbox{statement} \]
\begin{verbatim}
    cps void cpc_yield();
    cps void cpc_done();
\end{verbatim}

\paragraph{\tt cpc\_yield} The function {\tt cpc\_yield} causes the
current continuation to be suspended, and placed at the end of the
queue of runnable continuations.  Control is passed back to the CPC
main loop.  This statement has no effect in detached state.\\
{\bf Experimental:} this function accepts an optionnal {\tt int mode}
argument but this is likely to change in a near future.  You shall not
rely on it.

\paragraph{\tt cpc\_done} The function {\tt cpc\_done} causes the
current continuation to be discarded, and control to be passed back to
the main CPC loop.

\paragraph{\tt cpc\_spawn} The {\tt cpc\_spawn} statement creates a new
attached thread that executes the argument to {\tt cpc\_spawn} and
places it at the end of the queue of runnable continuations.  If this
argument contains free variables, they are handled like free variables
in CPS functions (Section~\ref{sec:cpc-functions}). Execution then
proceeds after the {\tt cpc\_spawn} statement (control is {\em not\/}
passed back to the main CPC loop).  This statement is valid in arbitrary
context.

\subsection{Synchronisation: condition variables}

\begin{verbatim}
    typedef struct cpc_condvar cpc_condvar;

    cpc_condvar *cpc_condvar_get(void);
    cpc_condvar *cpc_condvar_retain(cpc_condvar*);
    void cpc_condvar_release(cpc_condvar*);
    int cpc_condvar_count(cpc_condvar*);

    cps int cpc_wait(cpc_condvar *cond);
    void cpc_signal(cpc_condvar *);
    void cpc_signal_all(cpc_condvar *);
\end{verbatim}

\paragraph{\tt cpc\_wait} The function {\tt cpc\_wait} places the
current continuation on the list of continuations waiting on the
condition variable passed as argument to {\tt cpc\_wait}.  Control is
passed back to the CPC loop.  This function accepts two additionnal
optionnal arguments, specifying a timeout in seconds and microseconds.  
See {\tt cpc\_sleep} below for more details.  This function returns {\tt
CPC\_CONDVAR} or {\tt CPC\_TIMEOUT}, depending on the event which woke
it up.  This function is only valid in attached state.

\paragraph{\tt cpc\_signal} The function {\tt cpc\_signal} causes the
first of the continuations waiting on the condition variable passed as
argument to be moved to the tail of the queue of runnable continuations.
Execution proceeds at the instruction following the call to {\tt
  cpc\_signal}.  This statement is valid in arbitrary context, but only in
attached state.

\paragraph{\tt cpc\_signal\_all} The function {\tt cpc\_signal\_all}
causes all of the continuations waiting on the condition variabled passed
as argument to be moved to the tail of the queue of runnable continuations.
This function guarantees that the continuations will be run in the order in
which they were suspended.  This statement is valid in arbitrary context,
but only in attached state.

\subsection{Sleeping}

\begin{verbatim}
    cps int cpc_sleep(int sec, int usec, cpc_condvar *cond);
    double cpc_gettime(void);
\end{verbatim}

\paragraph{\tt cpc\_sleep} The function {\tt cpc\_sleep} takes three
arguments: a time in seconds, a time in microseconds, and a condition
variable.  It causes the current thread to be suspended until
either the specified amount of time has passed, or the condition
variable is signalled, whichever happens first.  It returns,
respectively, {\tt CPC\_TIMEOUT} or {\tt CPC\_CONDVAR}.

The third argument can be omitted if no interruption is necessary.
The second argument can be omitted if sub-second accuracy is not
needed.

The form with a third argument is only valid in attached state.

The following calls:
\begin{verbatim}
    cpc_wait(cond, sec);
    cpc_wait(cond, sec, usec);
\end{verbatim}
are syntactic sugar for, respectively,
\begin{verbatim}
    cpc_sleep(sec, 0, cond);
    cpc_sleep(sec, usec, cond);
\end{verbatim}

\paragraph{\tt cpc\_gettime} The function {\tt cpc\_gettime} returns a double
representing the current time in seconds.  This function is not valid in
detached mode.

\subsection{Waiting for I/O}

\begin{verbatim}
    cps int cpc_io_wait(int fd, int direction, cpc_condvar *cond);
    void cpc_signal_fd(int fd, int direction);
\end{verbatim}

\paragraph{\tt cpc\_io\_wait} The function {\tt cpc\_io\_wait} takes three
arguments: a file descriptor, a direction, and a condition variable.
The direction can be one of {\tt CPC\_IO\_IN}, meaning input, or {\tt
  CPC\_IO\_OUT}, meaning output.

This function causes the current continuation to be suspended until
either the given file descriptor is available for I/O in the given
direction, or the given condition variable is signalled, whichever
happens first.  It returns {\tt CPC\_IO\_IN}, {\tt CPC\_IO\_OUT} or {\tt
CPC\_CONDVAR}, depending on what happened.  It may also return {\tt -1}
if some error occurred (e.g. in detached mode, when the call to
{\tt poll} fails).

The form with a third argument is only valid in attached state.

\paragraph{\tt cpc\_signal\_fd} The function {\tt cpc\_signal\_fd} wakes
up the continuations waiting for an event {\tt direction} on the {\tt
fd} file descriptor.  It shall only be called in the same thread as the
CPC scheduler (but not necessarily in cps context) and will only wake up
attached threads. {\bf Not implemented in the current runtime!}

\section{Limitations and implementation notes}

Not all legal C code is allowable in CPC.  Some of the limitations
described below are fundamental to the implementation technique of
CPC; others are just artefacts of the current implementation, and will
be lifted in a future version.

\subsection{Fundamental limitations}

The use of the {\tt longjmp} library function, and its variants, is
not allowed in CPC code.

\subsection{Current limitations}

The current implementation is based on CIL.  The CIL limitations
therefore apply:
\begin{quote}
There are several implementation details of CIL that might make it
unusable or less than ideal for certain tasks:
\begin{itemize}
\item CIL operates after preprocessing. If you need to see comments, for
example, you cannot use CIL. But you can use attributes and pragmas
instead. And there is some support to help you patch the include
files before they are seen by the preprocessor. For example, this is
how we turn some {\tt \#define}s that we don’t like into function calls.
\item CIL does transform the code in a non-trivial way. This is done
in order to make most analyses easier. But if you want to see
the code {\tt e1, e2++} exactly as it appears in the code, then you
should not use CIL.
\item CIL removes all local scopes and moves all variables to
function scope. It also separates a declaration with an
initializer into a declaration plus an assignment. The
unfortunate effect of this transformation is that local
variables cannot have the {\tt const} qualifier.
\end{itemize}
From \url{http://hal.cs.berkeley.edu/cil/cil011.html}.
\end{quote}

\subsection{Time complexity of CPC operations}

The current implementation of CPC implements all the CPS operations in
constant time, with the following exceptions:\marginpar{\small This
might not be true anymore, but the old behaviour will eventually be
restored.}
\begin{itemize} 
\item when a continuation is queued on two structures simultaneously
  (because of {\tt cpc\_sleep} or {\tt cpc\_io\_wait} with a non-null
  last argument), invoking it requires dequeueing it from the second
  queue, which takes linear time in the worst case;
\item the {\tt cpc\_sleep} instruction runs in worst-case time
  proportional to the number of currently sleeping continuations;
\end{itemize}

\chapter{The CPC library}

The functions in the CPC library are themselves written in CPC, using
only the primitives documented in Chapter~\ref{chapter:language}.

All the functions in the CPC core library are declared in the file
{\tt cpc-lib.h}.

\section{Barriers}

\begin{verbatim}
    typedef struct cpc_barrier cpc_barrier;

    cpc_barrier *cpc_barrier_get(int count);
    cps void cpc_barrier_await(cpc_barrier *barrier);
\end{verbatim}

A barrier is a synchronisation construct that allows a set of
continuations to be woken up at the same time.  A barrier is
conceptually a queue of continuations and a count of continuations
remaining to wait for.

\paragraph{\tt cpc\_barrier\_get} The function {\tt cpc\_barrier\_get}
returns a new barrier initialised to wait for {\tt count}
continuations.

\paragraph{\tt cpc\_barrier\_await} The function {\tt cpc\_barrier\_await}
causes the current continuation to wait on the barrier given in
argument.  This function first decrements the barrier's count; if the
count reaches zero, it wakes up all of the continuations waiting on
the barrier.  Otherwise, it suspends the current continuation.

The function {\tt cpc\_barrier\_await} guarantees that the
continuations are run in the order in which they were suspended.

\section{Input/Output}

\subsection{Setting up file descriptors}

\begin{verbatim}
    int cpc_setup_descriptor(int fd, int nonagle);
\end{verbatim}

The function \verb|cpc_setup_descriptor| sets up the file descriptor
\verb|fd| into non-blocking mode, making it suitable for use by the
CPC runtime.  If \verb|nonagle| is true (non-zero), the descriptor is
assumed to refer to a socket and has the Nagle algorithm disabled (the
socket option \verb|TCP_NODELAY| is set).

This function returns 1 in case of success, -1 in case of failure.

\subsection{Input/Output}
\begin{verbatim}
  cps int cpc_write(int fd, void *buf, size_t count);
  cps int cpc_write_timeout(int fd, void *buf, size_t count,
                            int secs, int micros);
  cps int cpc_read(int fd, void *buf, size_t count);
  cps int cpc_read_timeout(int fd, void *buf, size_t count,
                           int secs, int micros);
\end{verbatim}

The functions \verb|cpc_write| and \verb|cpc_read| are CPC's versions
of the \verb|write| and \verb|read| system calls.  They return the
number of octets read/written in case of success; in case of failure,
they return -1 with \verb|errno| set.

The versions with \verb|timeout| appended return after \verb|secs|
seconds and \verb|micros| micreseconds if no I/O has been possible.
In this case, they return -1 with \verb|errno| set to \verb|EAGAIN|.

\begin{thebibliography}{MTHM90}

\bibitem[ISO99]{iso:c99}
Information technology --- programming language {C}.
International standard ISO/IEC~9899:1999, 1999.

\bibitem[Plo75]{plotkin:call-by-lambda}
G.~D. Plotkin.
Call-by-name, call-by-value and the lambda-calculus.
{\em Theoretical Computer Science}, 1:125--159, 1975.
Also published as Memorandum SAI--RM--6, School of Artificial
  Intelligence, University of Edinburgh, Edinburgh, 1973.

\bibitem[SW74]{strachey:continuations}
Christopher Strachey and Christopher P. Wadsworth.
Continuations: A mathematical semantics for handling full jumps.
Technical Monograph PRG--11, Oxford University Computing
Laboratory, Programming Research Group, Oxford, England, 1974.

\end{thebibliography}

\end{document}
